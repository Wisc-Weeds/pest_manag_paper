% Options for packages loaded elsewhere
\PassOptionsToPackage{unicode}{hyperref}
\PassOptionsToPackage{hyphens}{url}
%
\documentclass[
  12pt,
  a4paper,
]{article}
\usepackage{amsmath,amssymb}
\usepackage{lmodern}
\usepackage{setspace}
\usepackage{ifxetex,ifluatex}
\ifnum 0\ifxetex 1\fi\ifluatex 1\fi=0 % if pdftex
  \usepackage[T1]{fontenc}
  \usepackage[utf8]{inputenc}
  \usepackage{textcomp} % provide euro and other symbols
\else % if luatex or xetex
  \usepackage{unicode-math}
  \defaultfontfeatures{Scale=MatchLowercase}
  \defaultfontfeatures[\rmfamily]{Ligatures=TeX,Scale=1}
  \setmainfont[]{Arial}
\fi
% Use upquote if available, for straight quotes in verbatim environments
\IfFileExists{upquote.sty}{\usepackage{upquote}}{}
\IfFileExists{microtype.sty}{% use microtype if available
  \usepackage[]{microtype}
  \UseMicrotypeSet[protrusion]{basicmath} % disable protrusion for tt fonts
}{}
\makeatletter
\@ifundefined{KOMAClassName}{% if non-KOMA class
  \IfFileExists{parskip.sty}{%
    \usepackage{parskip}
  }{% else
    \setlength{\parindent}{0pt}
    \setlength{\parskip}{6pt plus 2pt minus 1pt}}
}{% if KOMA class
  \KOMAoptions{parskip=half}}
\makeatother
\usepackage{xcolor}
\IfFileExists{xurl.sty}{\usepackage{xurl}}{} % add URL line breaks if available
\IfFileExists{bookmark.sty}{\usepackage{bookmark}}{\usepackage{hyperref}}
\hypersetup{
  pdftitle={Off-target spray particle movement},
  pdfauthor={Rodrigo Werle; Bruno C Vieira; Maxwel C. Oliveira; Guilherme Alves; Greg Kruger},
  hidelinks,
  pdfcreator={LaTeX via pandoc}}
\urlstyle{same} % disable monospaced font for URLs
\usepackage[margin = 2.54cm]{geometry}
\usepackage{graphicx}
\makeatletter
\def\maxwidth{\ifdim\Gin@nat@width>\linewidth\linewidth\else\Gin@nat@width\fi}
\def\maxheight{\ifdim\Gin@nat@height>\textheight\textheight\else\Gin@nat@height\fi}
\makeatother
% Scale images if necessary, so that they will not overflow the page
% margins by default, and it is still possible to overwrite the defaults
% using explicit options in \includegraphics[width, height, ...]{}
\setkeys{Gin}{width=\maxwidth,height=\maxheight,keepaspectratio}
% Set default figure placement to htbp
\makeatletter
\def\fps@figure{htbp}
\makeatother
\setlength{\emergencystretch}{3em} % prevent overfull lines
\providecommand{\tightlist}{%
  \setlength{\itemsep}{0pt}\setlength{\parskip}{0pt}}
\setcounter{secnumdepth}{5}
\usepackage[left]{lineno}
\linenumbers
\usepackage{booktabs}
\usepackage{longtable}
\usepackage{array}
\usepackage{multirow}
\usepackage{wrapfig}
\usepackage{float}
\usepackage{colortbl}
\usepackage{pdflscape}
\usepackage{tabu}
\usepackage{threeparttable}
\usepackage{threeparttablex}
\usepackage[normalem]{ulem}
\usepackage{makecell}
\usepackage{xcolor}
\ifluatex
  \usepackage{selnolig}  % disable illegal ligatures
\fi
\newlength{\cslhangindent}
\setlength{\cslhangindent}{1.5em}
\newlength{\csllabelwidth}
\setlength{\csllabelwidth}{3em}
\newenvironment{CSLReferences}[2] % #1 hanging-ident, #2 entry spacing
 {% don't indent paragraphs
  \setlength{\parindent}{0pt}
  % turn on hanging indent if param 1 is 1
  \ifodd #1 \everypar{\setlength{\hangindent}{\cslhangindent}}\ignorespaces\fi
  % set entry spacing
  \ifnum #2 > 0
  \setlength{\parskip}{#2\baselineskip}
  \fi
 }%
 {}
\usepackage{calc}
\newcommand{\CSLBlock}[1]{#1\hfill\break}
\newcommand{\CSLLeftMargin}[1]{\parbox[t]{\csllabelwidth}{#1}}
\newcommand{\CSLRightInline}[1]{\parbox[t]{\linewidth - \csllabelwidth}{#1}\break}
\newcommand{\CSLIndent}[1]{\hspace{\cslhangindent}#1}

\title{Off-target spray particle movement}
\author{Rodrigo Werle \and Bruno C Vieira \and Maxwel C.
Oliveira \and Guilherme Alves \and Greg Kruger}
\date{2021-08-04}

\begin{document}
\maketitle
\begin{abstract}
BACKGROUND:
\end{abstract}

\setstretch{2}
\hypertarget{introduction}{%
\section{Introduction}\label{introduction}}

The widespread cases of glyphosate-resistant weeds throughout the US
motivated farmers to adopt dicamba and 2,4-D tolerant
crops.\textsuperscript{1--5} As expected, the agricultural usage of
dicamba and 2,4-D significantly increased in the US (``United States
Department of Agriculture'' 2020). Applications of dicamba and 2,4-D in
new cropping areas during extended time in the growing season has led to
increased opportunities for the off-target movement of these
herbicides.\textsuperscript{6,7} Spray drift is one of the main
mechanisms of pesticide off-target movement and can be defined as the
part of the pesticide application deflected away from the target area
during or following applications.\textsuperscript{8} Herbicide drift
potential is influenced by application technique, environmental
conditions, and surrounding vegetation.\textsuperscript{9--13}

Dicamba and 2,4-D spray drift is a concern because these herbicides are
very active on broadleaf vegetation, including soybean and cotton,
causing distinctive crop injury and yield loss at very low dose
exposures.\textsuperscript{14} The adoption of best management practices
during herbicide applications, including proper nozzle selection and the
use of drift reducing adjuvants, are among the main strategies to manage
spray droplet size and mitigate spray drift.\textsuperscript{15--17} New
dicamba and 2,4-D products are mostly applied using venturi nozzles with
air-inclusion and pre-orifice components that increase spray droplet
size and reduce spray drift potential.\textsuperscript{18,19} Despite
the advances in nozzle technology,\textsuperscript{19} pesticide
formulations and adjuvants,\textsuperscript{17} spraying
techniques,\textsuperscript{20} strategies to mitigate spray
drift,\textsuperscript{11} and education and extension
efforts,\textsuperscript{4,21} dicamba and 2,4-D spray drift remains
associated with crop injury complaints across the US. In addition to
this, common practices to mitigate spray drift, such as the use of DRAs
with venturi nozzles in combination with lower pressures and boom
heights, could potentially compromise spray uniformity, deposition, and
weed control\textsuperscript{22} (Moraes et al.~2021).

The use of spray shields or hoods is another strategy to mitigate spray
drift during pesticide applications.\textsuperscript{22--26} These
devices generally reduce spray drift potential by minimizing spray
exposure to wind. Foster et al.~(2018)\textsuperscript{24} reported that
a hooded sprayer reduced spray drift potential in approximately 50\%
regardless of the nozzles used compared to conventional applications.
The use of spray hoods can potentially extend the available time for
applications during windy conditions and allow the use of finer droplets
with lower carrier volumes to optimize spray coverage and weed
control.\textsuperscript{22} Although the benefits of sprayer hoods have
been studies and reported since the early fifties, the technique
adoption has been relatively low among farmers and
applicators.\textsuperscript{13} With the increased usage of dicamba and
2,4-D and the spray drift incidents associated with both herbicides,
crop protection professionals renewed their interest in spray hoods to
further mitigate herbicide off-target movement during applications.
Therefore, the objective of this study was to evaluate the effectiveness
of spray hoods in reducing pesticide drift for spray solutions and
nozzles typically used for new auxinic herbicide products.

\hypertarget{materials-and-methods}{%
\section{Materials and Methods}\label{materials-and-methods}}

\hypertarget{field-study}{%
\subsection{Field study}\label{field-study}}

Field experiments were conducted in three locations: North Platte, NE
(geographic coordinates) on September 11-12/2021; Arlington, WI
(geographic coordinates) on September 15-16/2021; and city, MO
(geographic coordinates) on September 18-20/2021. Applications were made
on bare ground, corn stubble (XX cm height) and ???? in NE, WI, and MO,
respectively. The fields were flat and with no surrounding trees.
Samples were processed and analyzed at the Pesticide Application
Technology Laboratory of the University of Nebraska-Lincoln located in
North Platte, Nebraska.

Three factors were evaluated: sprayer type, nozzle type, and spray
solution. Each experimental treatment was replicated 10 times. Two
sprayers (Manufacturer, city, state) were employed for this study, the
only difference being the inclusion of a hood or no hood. These sprayers
were 9.1 meters in width and each had a 1130 L polyethylene tank. Spray
delivery was accomplished via a hydraulic pump driven by the
accompanying tractor. Each sprayer was connected to its own tractor via
the three-point hitch system. Nozzle spacing was 51 cm, and boom height
was set at 90 cm above the ground level for both sprayers. The wind
skirt on the hooded sprayer was set slightly above the soil surface or
corn stubble. The height for each sprayer was maintained throughout the
study via the sprayers' guide wheels and the tractors' hitch system. The
hood was constructed of molded, polymer plastic that surrounded the
nozzles (Manufacturer, city, state). The hood sections reached
approximately 30.5 cm below the nozzle orifices, and a plastic curtain
reached a further 10 cm below the plastic hood. Nozzles were properly
attached to the boom in order to get no interference of hood and nozzle
plume.

Nozzle types were TTI11003 and AIXR11003 (Teejet Technologies, Wheaton,
IL), and ULD12003 (Pentair, Minneapolis, MN). All nozzles have air
inclusion features and the carrier volume was 140 L ha\^{}-1 applied at
276 kPa operating pressure and 2.6 m s-1 application speed.

The spray solutions were prepared at the same day of applications and
were water alone and water plus a drift-reducing adjuvant based on
polyethylene glycol, choline chloride, and guar gum (IntactTM, Precision
Laboratories, LLC, Waukegan, IL) at a rate of 0.5\% v v-1. Additionally,
a rhodamine fluorescent dye (Red Dye, Cole-Parmer, Vernon Hills, IL) was
added to all solutions at a rate of 0.5\% v v-1. The volumetric median
diameter (VMD) of droplets and volume percentage of droplets finer than
200 μm (V200) were measured at the Pesticide Application Technology
Laboratory using a Sympatec Helos-Vario K/R laser diffraction system
(Sympatec Inc., Clausthal, Germany), setup with a R7 lens, with a
dynamic size range of 9 to 1,800 μm. The distance from the nozzle tip to
the laser was 0.3 m. The VMD and V200 for each combination spray
solution versus nozzle type are listed in Table 1.

\hypertarget{application-and-field-layout}{%
\subsubsection{Application and field
layout}\label{application-and-field-layout}}

Prior to the applications, 27 drift collection stations were placed
downwind of the sprayed area in three transects (spaced by 7.6 m) and
perpendicular to the spray line (Figure 1). For each transect,
collectors were positioned at 1, 2, 3, 4, 8, 16, 31, 45, 60 m from the
edge of the application zone. Additionally, three drift stations were
placed 5 m upwind from the edge of the application zone and four petri
dishes (150 mm diameter) were placed in-swath. All collectors were
positioned 10 cm above the ground surface. Mylar cards (Grafix Plastics,
Cleveland, OH) were used as drift collectors. Cards with dimensions of
10 x 10 cm were placed upwind and up to 31 m downwind, whereas cards
with dimensions of 20 x 20 cm were placed at 45 and 60 m downwind.

Each replication was considered as one pass of the sprayer, equivalent
to 828 m2 in total area (91 m length x 9.1 m wide). Before each pass, a
new set of mylar cards and petri dishes were placed at the sampling
points. Five minutes after the end of each application was performed,
cards and petri dishes were collected and placed individually into
pre-labeled plastic bags. All samples were carefully managed to avoid
cross-contamination and stored into dark containers until further
analysis in laboratory in order to prevent photodegradation of rhodamine
dye. Samples were collected from the furthest to the nearest downwind
distance. Different teams were designated to work at downwind, upwind,
and in-swath zones, and gloves were changed between application passes.

The targeted wind velocity was between 3.6 to 6.7 m s-1 and ± 30° of
being perpendicular to the driveline before applying a treatment. When
necessary, the driveline and application zone was shifted to maintain
the ± 30° wind direction target. Meteorological conditions (air
temperature, and relative humidity, wind speed, wind direction) were
collected at 2 m height and 1-min intervals using a HOBO RX3000 Weather
Station (Onset Computer Co., Bourne, MA, USA) positioned upwind of the
sprayed area. The wind speed and direction data were collected using 2D
WindSonic anemometers (Gill Instruments, Lymington, UK). The
meteorological data for each respective treatment is listed in Figure 2.

\hypertarget{dye-quantification}{%
\subsubsection{Dye quantification}\label{dye-quantification}}

Samples were taken to the laboratory for dye extraction and
quantification using fluorometry technique. Distilled water was used as
extraction solution added to each bag using a bottle top dispenser
(LabSciences Inc., 60000-BTR, Reno, NV). Samples collected downwind were
rinsed with 50 mL of distilled water, whereas samples collected upwind
and in-swath were rinsed with 20 mL of distilled water. The samples were
agitated for 15 s and then a 1.5 mL aliquot from each sample bag was
drawn to fill a glass cuvette. The cuvette was placed in a rhodamine
module inside a fluorometer (Trilogy 7200.000, Turner Designs,
Sunnyvale, CA) using green light. Serial dilutions were performed upon
each tank sample to generate calibration curves, which allowed the
conversion of relative fluorescence unit into mg L-1 and further
expressing data into ɳL cm-2.

\hypertarget{greenhouse-study}{%
\subsection{Greenhouse study}\label{greenhouse-study}}

A completely randomized design

\hypertarget{statistical-analyses}{%
\subsection{Statistical analyses}\label{statistical-analyses}}

The statistical analyses were conducted with R statistical software
version 4.1.0.\textsuperscript{27} Data analyses were performed with
Bayesian inference with ``brms'' package.\textsuperscript{28} Bayesian
inference uses Markov chain Monte Carlo algorithms for sampling a
probability distribution;\textsuperscript{28} and avoids singular fit
from frequentist linear models when using complex random effects.

\hypertarget{field-study-1}{%
\subsubsection{Field study}\label{field-study-1}}

Solution, sprayer and nozzle factors were grouped as a single fixed
effect (herein treatments) due to missing factor water in Missouri.
Resulting in a combination of 12 treatments.

\hypertarget{spray-solution-deposition-at-upwind-and-inswath}{%
\paragraph{Spray solution deposition at upwind and
inswath}\label{spray-solution-deposition-at-upwind-and-inswath}}

Data was fitted to a mixed model using \emph{brm} function. Treatments
were the fixed effects and blocks nested within location random effects.
Model family was gaussian and prior distribution was set to student-t
with mean 0.5, standard deviation 3 and 11 degrees of freedom. The
posterior summaries (mean and highest posterior density) were estimated
with \emph{emmeans} function from the ``emmeans'' package. Treatment
means were compared using Bayes Factor (BF).\textsuperscript{29,30} In
short, if BF \textgreater{} 1 there is evidence for H1 (difference
between treatments); whereas, if H0 \textless{} 1 there is evidence for
H0 (no difference between treatments). If BF = 1, there is no evidence.
The level of evidence (anecdotal, moderate, strong, very strong, and
extreme) varies as the BF value increases (evidence for H1) or decreases
(evidence for H0).

\hypertarget{spray-solution-deposition-at-downwind}{%
\paragraph{Spray solution deposition at
downwind}\label{spray-solution-deposition-at-downwind}}

Data was fitted to a Bayesian linear mixed model using \emph{brm}
function. Spray solution deposition and distance were log-transformed to
meet linearity. A single model was fitted to each treatment. For each
model, treatments and distance were the fixed effects and blocks nested
within location random effects. Model family was gaussian and prior
distribution was set to student-t with mean 0.5, standard deviation 3
and 11 degrees of freedom. For clarification, intercepts, slopes were
back-transformed with \emph{exponential} function. Moreover, the linear
models fitted were used to predict the distance where no spray particle
deposition was detected (0 ηL cm\textsuperscript{-2}) for each
treatment, which was also back-transformed to m scale.

The area under the curve (AUC) was used to validate the linear models.
The spray solution deposition across distances within an experimental
unit were used to calculate the absolute AUC value. The AUC was
performed with \emph{audps} function from the ``agricolae'' package. The
AUC is commonly used for plant disease progress\textsuperscript{31,32}
but has been used to calculate herbicide injury.\textsuperscript{33}
Data was fitted to a mixed model using \emph{brm} function. Treatments
were the fixed effects and blocks nested within location random effects.
Model family was gaussian and prior distribution was set to student-t
with mean 0.5, standard deviation 3 and 11 degrees of freedom. The
posterior summaries and treatment means were estimated and compared
using Bayes Factor (BF) as above-mentioned.

\hypertarget{greenouse-study}{%
\subsection{Greenouse study}\label{greenouse-study}}

The Dv(10,50,90), RS, and \% drifable fines was fitted to a Bayesian
linear mixed model using \emph{brm} function. In the models, solution
and nozzle were set as fixed effects. Model family was hurdle gamma and
prior distribution was set to student-t with mean 0.5, standard
deviation 1 and 2 degrees of freedom. For each response variable, two
models were fitted: with and without interaction (solution and nozzle).
A model comparison was made with \textbf{bayesfactor\_models} from
``bayestestR'' package\textsuperscript{34} to investigate interaction
significance. For all response variables (Dvs, RS and \% drifable
fines), the best model was with interaction. The posterior summaries and
treatment means were estimated and compared using Bayes Factor (BF) as
above-mentioned.

\hypertarget{results}{%
\section{Results}\label{results}}

\hypertarget{spray-solution-deposition-at-inswath-and-upwind}{%
\subsection{Spray solution deposition at inswath and
upwind}\label{spray-solution-deposition-at-inswath-and-upwind}}

In general, Open sprayer treatments resulted in a more variable spray
particle deposition than Hood treatments (Figure 1). The inclusion of
either DRA or Water strongly impacted spray particle deposition inswath
for Open sprayer treatments, regardless nozzle type. The top and bottom
three treatments contained either DRA or Water, respectively. For
example, treatment DRA-Open-ULD resulted in the highest spray particle
deposition (1318.5 ηL cm\textsuperscript{-2}, Figure 1). In contrast,
911.2 nL cm\textsuperscript{-2} was the lowest spray solution
deposition, which was achieved with Water-Open-ULD treatment. Hood
sprayer treatments resulted in a more uniform spray particle deposition.
Furthermore, there were less than 0.29 ηL cm\textsuperscript{-2} spray
deposition at upwind with strong evidence (BF \textless{} 0.25) of no
difference between all pairwise treatment contrasts (data not shown).

\hypertarget{spray-solution-deposition-at-downwind-1}{%
\subsection{Spray solution deposition at
downwind}\label{spray-solution-deposition-at-downwind-1}}

Treatments with highest intercepts, which is the amount of spray
particle deposition near the treated area, were Water-Open-AIXR (15.7 ηL
cm\textsuperscript{-2}), followed by DRA-Open-AIXR (15.7 ηL
cm\textsuperscript{-2}), and Water-Open-ULD (12.0 ηL
cm\textsuperscript{-2}; Figure 2). In contrast, DRA-Hood-TTI,
DRA-Hood-ULD and DRA-Hood-AIXR treatments resulted in the lowest
intercepts (\textless{} 2.0 ηL cm\textsuperscript{-2}). In addition,
there is evidence that treatments with Hood sprayer provided faster
decay of spray particle deposition (slopes; Figure 2). The treatments
with highest slope decay were Water-Hood-TTI (-0.50), DRA-Hood-TTI
(-0.48), DRA-Hood-ULD (-0.44), Water-Hood-AIXR (-0.43), Water-Hood-ULD
(-0.43), and DRA-Hood-AIXR (-0.39).

The predicted distance where no spray particle deposition was detected
varied upon treatments (Figure 3A). In general, Open sprayer treatments
resulted in spray particle deposition at longest distances. For example,
the distance of non-detectable spray particle deposition with Open
sprayer treatments varied from 9.9 m (DRA-Open-TTI) to 54.9 m with
DRA-Open-AIXR; whereas Hood sprayer treatments varied from 1.4 to 8.2 m
with DRA-Hood-TTI and Water-Hood-AIXR, respectively.

Similar trend was observed in AUC. Treatments with Hood or Open sprayer
strongly impacted on AUC values (Figure 3B). The highest AUC values were
Water-Open-AIXR (87.4), followed by Water-Open-ULD (72.9) and
DRA-Open-AIXR (72.6). In contrast, DRA-Hood-TTI (13.9), DRA-Hood-ULD
(14.3) and Water-Open-AIXR (21.9) resulted in lowest AUC values. The
impact of Open and Hood sprayer is demonstrated in treatments including
AIXR nozzles. There is a high difference in AUC (50.7) between
DRA-Open-AIXR vs DRA-Hood-AIXR (BF \textgreater{} 100). Moreover,
addition of DRA did reduced AUC values when comparing within fixed
factors, sprayer (Hood and Open) and nozzle (AIXR, TTI and ULD);
however, addition of DRA were not statistically different for some
contrasts, including DRA-Hood-TTI vs Water-Hood-TTI (BF = 0.91),
DRA-Open-TTI vs Water-Open-TTI (BF = 0.55), and DRA-Open-AIXR vs
Water-Open-AIXR (BF = 0.93).

\hypertarget{dv-relative-span}{%
\subsection{Dv, Relative Span}\label{dv-relative-span}}

\hypertarget{discussion}{%
\section{Discussion}\label{discussion}}

\hypertarget{droplet-size}{%
\subsection{Droplet size}\label{droplet-size}}

As expected, the TTI nozzle produced coarser spray classification
compared to the ULD and the AIXR nozzles across all spray solutions
tested. In a response surface modeling approach in which nozzles were
tested at different operational setups, TTI nozzles had greater
percentage of operational setups producing Ultra Coarse (90\%) spray
classification compared to ULD (2\%) and AIXR (0\%)
nozzles.\textsuperscript{35} Legleiter et al.~(2018)\textsuperscript{36}
reported that glyphosate + 2,4-D (premixture formulation) applications
using a TTI11004 nozzle produced coarser spray classification (Ultra
Coarse) when compared to AIXR11004 (Very Coarse), TT11004 (Coarse), and
XR11004 (Medium) nozzles. Alves et al.~(2020)\textsuperscript{37}
reported that applications of different dicamba formulations using a
TTI110015 nozzle had larger droplet size across all formulations when
compared to applications using an AIXR110015 nozzle. Zaric
(2020)\textsuperscript{38} reported that the TTI11004 nozzle
consistently produced coarser spray droplet size across multiple spray
solutions when compared to the ULD12004 nozzle.

Creech et al.~(2015)\textsuperscript{18} reported that nozzle design had
major effects on spray droplet size, whereas spray solution influenced
droplet spectrum to a less extent. The addition of DRA increased spray
droplet size across all nozzles tested herein. Samples et
al.~(2021)\textsuperscript{39} reported that the addition of DRAs
(polymer and guargum) increased the spray droplet size of herbicide
spray mixtures containing combinations of glyphosate, glufosinate,
acetochlor, \emph{S}-metolachlor, dicamba, fomesafen, and/or 2,4-D
applied with an AIXR11005 nozzle. Zaric (2020)\textsuperscript{38}
reported that the addition of DRAs consistently increased spray droplet
size for dicamba applications using TTI1104 and ULD12004 spray nozzles.
These findings highlight the specific instructions regarding nozzle
selection for the new dicamba products (``Engenia Product Label'' 2021,
``Xtendimax Product Label'' 2021), in which nozzles producing Ultra
Coarse spray classifications are approved for applications. Label
instructions for new 2,4-D formulations are more flexible in terms of
nozzle selection, where Coarse, Very Coarse, and Extremely Coarse spray
classifications are allowed (``Enlist Duo Product Label'' 2021, ``Enlist
One Product Label'' 2021). New dicamba and 2,4-D product labels also
include specific DRAs to be used, especially when products are tank
mixed with other pesticides.

\hypertarget{spray-solution-deposition-in-swath}{%
\subsection{Spray solution deposition
in-swath}\label{spray-solution-deposition-in-swath}}

For conventional applications (open) without DRA, the TTI nozzle had
greater in-swath spray deposition volume on petri dishes compared to the
ULD and AIXR nozzles. It has been previously reported that applications
with Ultra Coarse spray had greater in-swath deposition volume on petri
dishes compared to finer spray classifications.\textsuperscript{40}
Finer sprays have a greater percentage of spray volume comprised of
small droplets that are more susceptible to spray
drift.\textsuperscript{19} These small droplets have more aerodynamic
drag and lose their initial momentum rapidly, therefore having their
trajectory highly influenced by the wind compared to coarser
droplets.\textsuperscript{41} It is important to highlight that greater
spray deposition volume collected on petri dishes does not mean greater
spray coverage, as coarser sprays result in reduced spray coverage when
compared to finer spray classifications.\textsuperscript{36,41}

For conventional applications (open), the addition of DRA consistently
increased in-swath spray deposition volume (petri dishes) compared to
applications without DRA. Samples et al.~(2021)\textsuperscript{39}
reported that the addition of DRAs (polymer and guargum) to several
herbicide tank mixtures had inconsistent deposition results on soybean
and cotton when compared to herbicide applications without adjuvants.
Authors reported that while the addition of polymer and guargam based
adjuvants increased cotton spray deposition for glyphosate + dicamba +
\emph{S}-Metolachlor spray mixture, the same adjuvants had no effect or
decreased cotton spray deposition when mixed to glyphosate + 2,4-D.
Moraes et al.~(2021) (Moraes et al.~2021) reported that the addition of
DRAs could have a negative effect on spray uniformity, especially with
nozzles producing coarser spray classifications at lower spray
pressures.

Results reported herein indicate an inconsistent effect of hooded
sprayers on in-swath spray deposition. For applications with DRA, the
conventional sprayer (open) had greater in-swath deposition compared to
hooded applications. On the other hand, for applications without DRA,
the hooded sprayer had greater in-swath deposition compared to the
conventional sprayer. Wolf et al.~(1993)\textsuperscript{22} observed
that solid spray shields generally decreased in-swath deposition
uniformity. Authors noted that spray solution dripping from sprayer
shield walls could increase the in-swath deposition variability during
applications. Wolf et al (1993)\textsuperscript{22} and Moraes et
al.~(2021) observations could explain the lower spray deposition
obtained for hooded applications with the addition of the DRA, although
this hypothesis needs to be further investigated.

\hypertarget{spray-solution-deposition-at-downwind-2}{%
\subsection{Spray solution deposition at
downwind}\label{spray-solution-deposition-at-downwind-2}}

For conventional (open) applications without DRA, the TTI nozzle had
less drift potential compared to AIXR and ULD nozzles, as expected.
Alves et al.~(2020)\textsuperscript{37} reported similar results in a
wind tunnel study, where applications of different dicamba formulations
using a TTI nozzle had less spray drift compared to applications using
an AIXR nozzle. Authors reported that soybean plants were still showing
50\% biomass reduction from 9.9 to 13.4 m downwind the nozzle for
applications with the AIXR nozzle, whereas this distance range was
reduced to 6.4 to 7.2 m for applications using the TTI nozzle. In
another wind tunnel study, Alves et al.~(2017)\textsuperscript{42}
reported that dicamba applications with a TTI nozzle had 0.6\% spray
drift at 12 m downwind the nozzle, whereas applications with XR, TT, and
AIXR nozzles had 13.5, 6.6, and 3.2\% drift, respectively. As previously
reported,\textsuperscript{17,43} the addition of DRA reduced spray drift
across all nozzles tested herein.

Hooded applications substantially reduced spray drift potential across
all treatment scenarios. For instance, hooded applications using the
AIXR nozzle without DRA had similar drift potential compared to
conventional applications using the TTI nozzle with DRA despite the
major droplet size differences between these treatments. These results
indicate that the adoption of spray hoods combined with proper nozzle
selection and the addition of DRAs to the spray solution can
substantially reduce spray drift potential during pesticide
applications. Foster et al.~(2018)\textsuperscript{24} reported that the
advent of spray hood considerably reduced spray drift for glyphosate
applications using Fine, Medium, Very Coarse, and Ultra Coarse spray
qualities. In a wind tunnel study, Ozkan et
al.~(1997)\textsuperscript{25} reported that several spray shields were
effective in reducing spray drift, where a double-foil shield reduced
spray drift in 59\% compared to applications without shields. In
laboratory tests, Smith et al.~(1982) reported that a spray shield could
reduce spray drift up to 70\%, although field studies indicated an
inconsistent performance where spray shields either increased or
decreased spray drift during tests.

The results reported in this study highlight the importance of adopting
best management practices during pesticide applications to mitigate
spray off-target movement to the surrounding environment. Hooded
applications with AIXR and ULD nozzles resulted in satisfactory spray
drift reduction despite the finer droplet spectrum compared to the TTI
nozzle. This could have implications for weed control as applications
with Ultra Coarse droplet spectrum have reduced spray coverage compared
to finer spray classifications. Legleiter et
al.~(2018)\textsuperscript{36} reported that glyphosate + 2,4-D
applications with Ultra Coarse (TTI11004) spray had reduced spray
coverage on artificial targets (cards) compared to Medium (XR11004),
Coarse (TT11004), and Very Coarse (AIXR11004) sprays. However, authors
reported that spray droplet spectrum did not influence 2,4-D absorption
in Palmer amaranth (\emph{Amaranthus palmeri}), waterhemp
(\emph{Amaranthus tuberculatus}), giant ragweed (\emph{Ambrosia
trifida}), and horseweed (\emph{Erigeron canadensis}). Increasing the
application carrier volume is one strategy to overcome the reduced
coverage from applications using coarser sprays. Butts et
al.~(2018)\textsuperscript{44} reported that dicamba applications with
Ultra Coarse spray resulted in optimal weed control for applications at
187 L ha-1, whereas Extremely Coarse spray resulted optimal weed control
for glufosinate applications. However, applications with increased spray
carrier volume are less efficient due to additional time and water
requirements during the operation.\textsuperscript{22} Hooded
applications using AIXR and ULD nozzles can provide increased spray
coverage compared to TTI nozzles while having low drift potential, which
allows for applications at reduced carrier volumes.

Because of the substantial drift reduction, applications with hooded
sprayers could also reduce product buffer zone requirements. New dicamba
formulations provide incentives to the use of qualified hooded sprayers
by decreasing the required downwind buffer from 73 m to 33 m (``Engenia
Product Label'' 2021, ``Xtendimax Product Label'' 2021). Although the
benefits associated with hooded sprayers, the technology has not been
well accepted by farmers. Nordby and Skuterud (1974)\textsuperscript{13}
mentioned that farmers could see spray hoods as cumbersome and expensive
devices that do not allow applicators to check nozzles during
applications. Considering the widespread adoption of dicamba and 2,4-D
tolerant crops and the increase in these herbicide usage, hooded
sprayers could be a very effective technique to mitigate spray drift
during pesticide applications.

\hypertarget{conclusion}{%
\section{Conclusion}\label{conclusion}}

\hypertarget{acknowledgments}{%
\section{Acknowledgments}\label{acknowledgments}}

\hypertarget{conflict-of-interest-declaration}{%
\section{Conflict of Interest
Declaration}\label{conflict-of-interest-declaration}}

\hypertarget{tables-each-table-complete-with-title-and-footnotes}{%
\section{Tables (each table complete with title and
footnotes)}\label{tables-each-table-complete-with-title-and-footnotes}}

\hypertarget{figure-legends}{%
\section{Figure Legends}\label{figure-legends}}

\hypertarget{references}{%
\section*{References}\label{references}}
\addcontentsline{toc}{section}{References}

\hypertarget{refs}{}
\begin{CSLReferences}{1}{0}
\leavevmode\hypertarget{ref-behrens2007}{}%
\CSLLeftMargin{1 }
\CSLRightInline{Behrens MR, Mutlu N, Chakraborty S, Dumitru R, Jiang WZ,
LaVallee BJ, \emph{et al.}, Dicamba {Resistance}: {Enlarging} and
{Preserving Biotechnology}-{Based Weed Management Strategies},
\emph{Science} \textbf{316}:1185--1188, {American Association for the
Advancement of Science} (2007).}

\leavevmode\hypertarget{ref-heap2018}{}%
\CSLLeftMargin{2 }
\CSLRightInline{Heap I and Duke SO, Overview of glyphosate-resistant
weeds worldwide, \emph{Pest Management Science} \textbf{74}:1040--1049
(2018).}

\leavevmode\hypertarget{ref-sammons2014}{}%
\CSLLeftMargin{3 }
\CSLRightInline{Sammons RD and Gaines TA, Glyphosate resistance: State
of knowledge, \emph{Pest Management Science} \textbf{70}:1367--1377
(2014).}

\leavevmode\hypertarget{ref-werle2018}{}%
\CSLLeftMargin{4 }
\CSLRightInline{Werle R, Oliveira MC, Jhala AJ, Proctor CA, Rees J, and
Klein R, Survey of {Nebraska Farmers}' {Adoption} of
{Dicamba}-{Resistant Soybean Technology} and {Dicamba Off}-{Target
Movement}, \emph{Weed Technol} \textbf{32}:754--761 (2018).}

\leavevmode\hypertarget{ref-wright2010}{}%
\CSLLeftMargin{5 }
\CSLRightInline{Wright TR, Shan G, Walsh TA, Lira JM, Cui C, Song P,
\emph{et al.}, Robust crop resistance to broadleaf and grass herbicides
provided by aryloxyalkanoate dioxygenase transgenes, \emph{Proceedings
of the National Academy of Sciences} \textbf{107}:20240--20245 (2010).}

\leavevmode\hypertarget{ref-soltaniOfftargetMovementAssessment2020}{}%
\CSLLeftMargin{6 }
\CSLRightInline{Soltani N, Oliveira MC, Alves GS, Werle R, Norsworthy
JK, Sprague CL, \emph{et al.}, Off-target movement assessment of dicamba
in {North America}, \emph{Weed Technol} \textbf{34}:318--330 (2020).}

\leavevmode\hypertarget{ref-werle2021}{}%
\CSLLeftMargin{7 }
\CSLRightInline{Werle R, Mobli A, Striegel S, Arneson N, DeWerff R,
Brown A, \emph{et al.}, Large {Scale Evaluation} of 2,4-{D Choline
Off}-target {Movement} and {Injury} in 2,4-{D}-susceptible {Soybean},
\emph{Weed Technology}:1--22, {Cambridge University Press} (2021).}

\leavevmode\hypertarget{ref-matthews2008}{}%
\CSLLeftMargin{8 }
\CSLRightInline{Matthews G, Pesticide {Application Methods}, {John Wiley
\& Sons} (2008).}

\leavevmode\hypertarget{ref-alheidary2014}{}%
\CSLLeftMargin{9 }
\CSLRightInline{Al Heidary M, Douzals JP, Sinfort C, and Vallet A,
Influence of spray characteristics on potential spray drift of field
crop sprayers: {A} literature review, \emph{Crop Protection}
\textbf{63}:120--130 (2014).}

\leavevmode\hypertarget{ref-arvidsson2011}{}%
\CSLLeftMargin{10 }
\CSLRightInline{Arvidsson T, Bergström L, and Kreuger J, Spray drift as
influenced by meteorological and technical factors, \emph{Pest
Management Science} \textbf{67}:586--598 (2011).}

\leavevmode\hypertarget{ref-felsot2010}{}%
\CSLLeftMargin{11 }
\CSLRightInline{Felsot AS, Unsworth JB, Linders JBHJ, Roberts G, Rautman
D, Harris C, \emph{et al.}, Agrochemical spray drift; assessment and
mitigation---{A} review*, \emph{Journal of Environmental Science and
Health, Part B} \textbf{46}:1--23, {Taylor \& Francis} (2010).}

\leavevmode\hypertarget{ref-hewitt2000}{}%
\CSLLeftMargin{12 }
\CSLRightInline{Hewitt AJ, Spray drift: Impact of requirements to
protect the environment, \emph{Crop Protection} \textbf{19}:623--627
(2000).}

\leavevmode\hypertarget{ref-nordby1974}{}%
\CSLLeftMargin{13 }
\CSLRightInline{Nordby A and Skuterud R, The effects of boom height,
working pressure and wind speed on spray drift, \emph{Weed Research}
\textbf{14}:385--395 (1974).}

\leavevmode\hypertarget{ref-egan2014}{}%
\CSLLeftMargin{14 }
\CSLRightInline{Egan JF, Barlow KM, and Mortensen DA, A
{Meta}-{Analysis} on the {Effects} of 2,4-{D} and {Dicamba Drift} on
{Soybean} and {Cotton}, \emph{Weed Science} \textbf{62}:193--206,
{Cambridge University Press} (2014).}

\leavevmode\hypertarget{ref-etheridge1999}{}%
\CSLLeftMargin{15 }
\CSLRightInline{Etheridge RE, Womac AR, and Mueller TC, Characterization
of the {Spray Droplet Spectra} and {Patterns} of {Four Venturi}-{Type
Drift Reduction Nozzles}, \emph{Weed Technology} \textbf{13}:765--770,
{Cambridge University Press} (1999).}

\leavevmode\hypertarget{ref-hewitt2008}{}%
\CSLLeftMargin{16 }
\CSLRightInline{Hewitt AJ, Spray optimization through application and
liquid physical property variables--{I}, \emph{Environmentalist}
\textbf{28}:25--30 (2008).}

\leavevmode\hypertarget{ref-hilz2013}{}%
\CSLLeftMargin{17 }
\CSLRightInline{Hilz E and Vermeer AWP, Spray drift review: {The} extent
to which a formulation can contribute to spray drift reduction,
\emph{Crop Protection} \textbf{44}:75--83 (2013).}

\leavevmode\hypertarget{ref-creech2015}{}%
\CSLLeftMargin{18 }
\CSLRightInline{Creech CF, Henry RS, Fritz BK, and Kruger GR, Influence
of {Herbicide Active Ingredient}, {Nozzle Type}, {Orifice Size}, {Spray
Pressure}, and {Carrier Volume Rate} on {Spray Droplet Size
Characteristics}, \emph{Weed Technology} \textbf{29}:298--310,
{Cambridge University Press} (2015).}

\leavevmode\hypertarget{ref-dorr2013}{}%
\CSLLeftMargin{19 }
\CSLRightInline{Dorr GJ, Hewitt AJ, Adkins SW, Hanan J, Zhang H, and
Noller B, A comparison of initial spray characteristics produced by
agricultural nozzles, \emph{Crop Protection} \textbf{53}:109--117
(2013).}

\leavevmode\hypertarget{ref-butts2019}{}%
\CSLLeftMargin{20 }
\CSLRightInline{Butts TR, Butts LE, Luck JD, Fritz BK, Hoffmann WC, and
Kruger GR, Droplet size and nozzle tip pressure from a pulse-width
modulation sprayer, \emph{Biosystems Engineering} \textbf{178}:52--69
(2019).}

\leavevmode\hypertarget{ref-butts2021}{}%
\CSLLeftMargin{21 }
\CSLRightInline{Butts TR, Barber LT, Norsworthy JK, and Davis J, Survey
of ground and aerial herbicide application practices in {Arkansas}
agronomic crops, \emph{Weed Technology} \textbf{35}:1--11, {Cambridge
University Press} (2021).}

\leavevmode\hypertarget{ref-wolf1993}{}%
\CSLLeftMargin{22 }
\CSLRightInline{Wolf TM, Grover R, Wallace K, Shewchuk SR, and Maybank
J, Effect of protective shields on drift and deposition characteristics
of field sprayers, \emph{Can J Plant Sci} \textbf{73}:1261--1273, {NRC
Research Press} (1993).}

\leavevmode\hypertarget{ref-burnside1968}{}%
\CSLLeftMargin{23 }
\CSLRightInline{Burnside OC, A {Shielded}, {Tractor}-{Mounted Sprayer}
for {Research Plots}, \emph{Weed Science} \textbf{16}:386--388,
{Cambridge University Press} (1968).}

\leavevmode\hypertarget{ref-foster2018}{}%
\CSLLeftMargin{24 }
\CSLRightInline{Foster HC, Sperry BP, Reynolds DB, Kruger GR, and
Claussen S, Reducing {Herbicide Particle Drift}: {Effect} of {Hooded
Sprayer} and {Spray Quality}, \emph{Weed Technology}
\textbf{32}:714--721, {Cambridge University Press} (2018).}

\leavevmode\hypertarget{ref-ozkan1997}{}%
\CSLLeftMargin{25 }
\CSLRightInline{Ozkan H, Miralles A, Sinfort C, Zhu H, and Fox R,
Shields to {Reduce Spray Drift}, \emph{Journal of Agricultural
Engineering Research} \textbf{67}:311--322 (1997).}

\leavevmode\hypertarget{ref-d.b.smith1982}{}%
\CSLLeftMargin{26 }
\CSLRightInline{D. B. Smith, F. D. Harris, and B. J. Butler, Shielded
{Sprayer Boom} to {Reduce Drift}, \emph{Transactions of the ASAE}
\textbf{25}:1136--1140 (1982).}

\leavevmode\hypertarget{ref-LanguageEnvironmentStatistical2021}{}%
\CSLLeftMargin{27 }
\CSLRightInline{R: {A Language} and {Environment} for {Statistical
Computing}, {R Foundation for Statistical Computing}, {Vienna, Austria}
(2021).}

\leavevmode\hypertarget{ref-burknerBrmsPackageBayesian2017}{}%
\CSLLeftMargin{28 }
\CSLRightInline{Bürkner P-C, \textbf{Brms} : {An} {\emph{R}} {Package}
for {Bayesian Multilevel Models Using} {\emph{Stan}}, \emph{J Stat Soft}
\textbf{80} (2017).}

\leavevmode\hypertarget{ref-leeBayesianCognitiveModeling2014}{}%
\CSLLeftMargin{29 }
\CSLRightInline{Lee MD and Wagenmakers E-J, Bayesian {Cognitive
Modeling}: {A Practical Course}, {Cambridge University Press} (2014).}

\leavevmode\hypertarget{ref-kassBayesFactors1995}{}%
\CSLLeftMargin{30 }
\CSLRightInline{Kass RE and Raftery AE, Bayes {Factors}, \emph{Journal
of the American Statistical Association} \textbf{90}:773--795,
{{[}American Statistical Association, Taylor \& Francis, Ltd.{]}}
(1995).}

\leavevmode\hypertarget{ref-meenaAreaDiseaseProgress2011}{}%
\CSLLeftMargin{31 }
\CSLRightInline{Meena PD, Chattopadhyay C, Meena SS, and Kumar A, Area
under disease progress curve and apparent infection rate of {Alternaria}
blight disease of {Indian} mustard ({Brassica} juncea) at different
plant age, \emph{Archives of Phytopathology and Plant Protection}
\textbf{44}:684--693, {Taylor \& Francis} (2011).}

\leavevmode\hypertarget{ref-simkoAreaDiseaseProgress2011}{}%
\CSLLeftMargin{32 }
\CSLRightInline{Simko I and Piepho H-P, The {Area Under} the {Disease
Progress Stairs}: {Calculation}, {Advantage}, and {Application},
\emph{Phytopathology®} \textbf{102}:381--389, {Scientific Societies}
(2011).}

\leavevmode\hypertarget{ref-striegelSpraySolutionPH2021}{}%
\CSLLeftMargin{33 }
\CSLRightInline{Striegel S, Oliveira MC, Arneson N, Conley SP,
Stoltenberg DE, and Werle R, Spray solution {pH} and soybean injury as
influenced by synthetic auxin formulation and spray additives,
\emph{Weed Technol} \textbf{35}:113--127 (2021).}

\leavevmode\hypertarget{ref-makowski2019}{}%
\CSLLeftMargin{34 }
\CSLRightInline{Makowski D, Ben-Shachar M, and Lüdecke D, {bayestestR}:
{Describing Effects} and their {Uncertainty}, {Existence} and
{Significance} within the {Bayesian Framework}, \emph{JOSS}
\textbf{4}:1541 (2019).}

\leavevmode\hypertarget{ref-fritz2016}{}%
\CSLLeftMargin{35 }
\CSLRightInline{Fritz B, Hoffman W, and Anderson J, Response {Surface
Method} for {Evaluation} of the {Performance} of {Agricultural
Application Spray Nozzles}, \emph{Pesticide Formulation and Delivery
Systems: 35th Volume, Pesticide Formulations, Adjuvants, and Spray
Characterization in 2014}, {ASTM International} (2016).}

\leavevmode\hypertarget{ref-legleiter2018}{}%
\CSLLeftMargin{36 }
\CSLRightInline{Legleiter TR, Young BG, and Johnson WG, Glyphosate plus
2,4-{D Deposition}, {Absorption}, and {Efficacy} on
{Glyphosate}-{Resistant Weed Species} as {Influenced} by {Broadcast
Spray Nozzle}, \emph{Weed Technology} \textbf{32}:141--149, {Cambridge
University Press} (2018).}

\leavevmode\hypertarget{ref-alves2020}{}%
\CSLLeftMargin{37 }
\CSLRightInline{Alves GS, Vieira BC, Ynfante RS, Santana TM, Moraes JG,
Golus JA, \emph{et al.}, Tank contamination and simulated drift effects
of dicamba-containing formulations on soybean cultivars,
\emph{Agrosystems, Geosciences \& Environment} \textbf{3}:e20065
(2020).}

\leavevmode\hypertarget{ref-zaric2020}{}%
\CSLLeftMargin{38 }
\CSLRightInline{Zaric M, Effects of {Tank Contamination} and {Impact} of
{Drift}-{Reducing Agents} on {Weed Control} in {Response} to {Dicamba
Applications}, \emph{Theses, Dissertations, and Student Research in
Agronomy and Horticulture} (2020).}

\leavevmode\hypertarget{ref-samples2021}{}%
\CSLLeftMargin{39 }
\CSLRightInline{Samples CA, Butts TR, Vieira BC, Irby JT, Reynolds DB,
Catchot A, \emph{et al.}, Effect of {Deposition Aids Tank}-{Mixed} with
{Herbicides} on {Cotton} and {Soybean Canopy Deposition} and {Spray
Droplet Parameters}, \emph{Agronomy} \textbf{11}:278, {Multidisciplinary
Digital Publishing Institute} (2021).}

\leavevmode\hypertarget{ref-vieira2018}{}%
\CSLLeftMargin{40 }
\CSLRightInline{Vieira BC, Butts TR, Rodrigues AO, Golus JA, Schroeder
K, and Kruger GR, Spray particle drift mitigation using field corn
({Zea} mays {L}.) As a drift barrier, \emph{Pest Management Science}
\textbf{74}:2038--2046 (2018).}

\leavevmode\hypertarget{ref-knoche1994}{}%
\CSLLeftMargin{41 }
\CSLRightInline{Knoche M, Effect of droplet size and carrier volume on
performance of foliage-applied herbicides, \emph{Crop Protection}
\textbf{13}:163--178 (1994).}

\leavevmode\hypertarget{ref-alves2017}{}%
\CSLLeftMargin{42 }
\CSLRightInline{Alves GS, Kruger GR, Cunha JPAR da, Santana DG de, Pinto
LAT, Guimarães F, \emph{et al.}, Dicamba {Spray Drift} as {Influenced}
by {Wind Speed} and {Nozzle Type}, \emph{Weed Technology}
\textbf{31}:724--731, {Cambridge University Press} (2017).}

\leavevmode\hypertarget{ref-johnson2006}{}%
\CSLLeftMargin{43 }
\CSLRightInline{Johnson AK, Roeth FW, Martin AR, and Klein RN,
Glyphosate {Spray Drift Management} with {Drift}-{Reducing Nozzles} and
{Adjuvants}, \emph{Weed Technology} \textbf{20}:893--897, {Cambridge
University Press} (2006).}

\leavevmode\hypertarget{ref-butts2018}{}%
\CSLLeftMargin{44 }
\CSLRightInline{Butts TR, Samples CA, Franca LX, Dodds DM, Reynolds DB,
Adams JW, \emph{et al.}, Spray droplet size and carrier volume effect on
dicamba and glufosinate efficacy, \emph{Pest Management Science}
\textbf{74}:2020--2029 (2018).}

\end{CSLReferences}

\end{document}
