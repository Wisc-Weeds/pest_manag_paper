% Options for packages loaded elsewhere
\PassOptionsToPackage{unicode}{hyperref}
\PassOptionsToPackage{hyphens}{url}
%
\documentclass[
  12pt,
  a4paper,
]{article}
\usepackage{amsmath,amssymb}
\usepackage{lmodern}
\usepackage{setspace}
\usepackage{ifxetex,ifluatex}
\ifnum 0\ifxetex 1\fi\ifluatex 1\fi=0 % if pdftex
  \usepackage[T1]{fontenc}
  \usepackage[utf8]{inputenc}
  \usepackage{textcomp} % provide euro and other symbols
\else % if luatex or xetex
  \usepackage{unicode-math}
  \defaultfontfeatures{Scale=MatchLowercase}
  \defaultfontfeatures[\rmfamily]{Ligatures=TeX,Scale=1}
  \setmainfont[]{Arial}
\fi
% Use upquote if available, for straight quotes in verbatim environments
\IfFileExists{upquote.sty}{\usepackage{upquote}}{}
\IfFileExists{microtype.sty}{% use microtype if available
  \usepackage[]{microtype}
  \UseMicrotypeSet[protrusion]{basicmath} % disable protrusion for tt fonts
}{}
\makeatletter
\@ifundefined{KOMAClassName}{% if non-KOMA class
  \IfFileExists{parskip.sty}{%
    \usepackage{parskip}
  }{% else
    \setlength{\parindent}{0pt}
    \setlength{\parskip}{6pt plus 2pt minus 1pt}}
}{% if KOMA class
  \KOMAoptions{parskip=half}}
\makeatother
\usepackage{xcolor}
\IfFileExists{xurl.sty}{\usepackage{xurl}}{} % add URL line breaks if available
\IfFileExists{bookmark.sty}{\usepackage{bookmark}}{\usepackage{hyperref}}
\hypersetup{
  pdftitle={Off-target spray particle movement},
  pdfauthor={Rodrigo Werle; Bruno C Vieira; Maxwel C. Oliveira; Guilherme Alves; Greg Kruger},
  hidelinks,
  pdfcreator={LaTeX via pandoc}}
\urlstyle{same} % disable monospaced font for URLs
\usepackage[margin = 2.54cm]{geometry}
\usepackage{graphicx}
\makeatletter
\def\maxwidth{\ifdim\Gin@nat@width>\linewidth\linewidth\else\Gin@nat@width\fi}
\def\maxheight{\ifdim\Gin@nat@height>\textheight\textheight\else\Gin@nat@height\fi}
\makeatother
% Scale images if necessary, so that they will not overflow the page
% margins by default, and it is still possible to overwrite the defaults
% using explicit options in \includegraphics[width, height, ...]{}
\setkeys{Gin}{width=\maxwidth,height=\maxheight,keepaspectratio}
% Set default figure placement to htbp
\makeatletter
\def\fps@figure{htbp}
\makeatother
\setlength{\emergencystretch}{3em} % prevent overfull lines
\providecommand{\tightlist}{%
  \setlength{\itemsep}{0pt}\setlength{\parskip}{0pt}}
\setcounter{secnumdepth}{5}
\usepackage[left]{lineno}
\linenumbers
\usepackage{booktabs}
\usepackage{longtable}
\usepackage{array}
\usepackage{multirow}
\usepackage{wrapfig}
\usepackage{float}
\usepackage{colortbl}
\usepackage{pdflscape}
\usepackage{tabu}
\usepackage{threeparttable}
\usepackage{threeparttablex}
\usepackage[normalem]{ulem}
\usepackage{makecell}
\usepackage{xcolor}
\ifluatex
  \usepackage{selnolig}  % disable illegal ligatures
\fi
\newlength{\cslhangindent}
\setlength{\cslhangindent}{1.5em}
\newlength{\csllabelwidth}
\setlength{\csllabelwidth}{3em}
\newenvironment{CSLReferences}[2] % #1 hanging-ident, #2 entry spacing
 {% don't indent paragraphs
  \setlength{\parindent}{0pt}
  % turn on hanging indent if param 1 is 1
  \ifodd #1 \everypar{\setlength{\hangindent}{\cslhangindent}}\ignorespaces\fi
  % set entry spacing
  \ifnum #2 > 0
  \setlength{\parskip}{#2\baselineskip}
  \fi
 }%
 {}
\usepackage{calc}
\newcommand{\CSLBlock}[1]{#1\hfill\break}
\newcommand{\CSLLeftMargin}[1]{\parbox[t]{\csllabelwidth}{#1}}
\newcommand{\CSLRightInline}[1]{\parbox[t]{\linewidth - \csllabelwidth}{#1}\break}
\newcommand{\CSLIndent}[1]{\hspace{\cslhangindent}#1}

\title{Off-target spray particle movement}
\author{Rodrigo Werle \and Bruno C Vieira \and Maxwel C.
Oliveira \and Guilherme Alves \and Greg Kruger}
\date{2021-07-15}

\begin{document}
\maketitle
\begin{abstract}
BACKGROUND:
\end{abstract}

\setstretch{2}
\hypertarget{introduction}{%
\section{Introduction}\label{introduction}}

Integrate pest management The off-target particle movement

The objective of this stu

\hypertarget{materials-and-methods}{%
\section{Materials and Methods}\label{materials-and-methods}}

\hypertarget{field-study}{%
\subsection{Field study}\label{field-study}}

Field experiments were conducted in three locations: North Platte, NE
(geographic coordinates) on September 11-12/2021; Arlington, WI
(geographic coordinates) on September 15-16/2021; and city, MO
(geographic coordinates) on September 18-20/2021. Applications were made
on bare ground, corn stubble (XX cm height) and ???? in NE, WI, and MO,
respectively. The fields were flat and with no surrounding trees.
Samples were processed and analyzed at the Pesticide Application
Technology Laboratory of the University of Nebraska-Lincoln located in
North Platte, Nebraska.

Three factors were evaluated: sprayer type, nozzle type, and spray
solution. Each experimental treatment was replicated 10 times. Two
sprayers (Manufacturer, city, state) were employed for this study, the
only difference being the inclusion of a hood or no hood. These sprayers
were 9.1 meters in width and each had a 1130 L polyethylene tank. Spray
delivery was accomplished via a hydraulic pump driven by the
accompanying tractor. Each sprayer was connected to its own tractor via
the three-point hitch system. Nozzle spacing was 51 cm, and boom height
was set at 90 cm above the ground level for both sprayers. The wind
skirt on the hooded sprayer was set slightly above the soil surface or
corn stubble. The height for each sprayer was maintained throughout the
study via the sprayers' guide wheels and the tractors' hitch system. The
hood was constructed of molded, polymer plastic that surrounded the
nozzles (Manufacturer, city, state). The hood sections reached
approximately 30.5 cm below the nozzle orifices, and a plastic curtain
reached a further 10 cm below the plastic hood. Nozzles were properly
attached to the boom in order to get no interference of hood and nozzle
plume.

Nozzle types were TTI11003 and AIXR11003 (Teejet Technologies, Wheaton,
IL), and ULD12003 (Pentair, Minneapolis, MN). All nozzles have air
inclusion features and the carrier volume was 140 L ha\^{}-1 applied at
276 kPa operating pressure and 2.6 m s-1 application speed.

The spray solutions were prepared at the same day of applications and
were water alone and water plus a drift-reducing adjuvant based on
polyethylene glycol, choline chloride, and guar gum (IntactTM, Precision
Laboratories, LLC, Waukegan, IL) at a rate of 0.5\% v v-1. Additionally,
a rhodamine fluorescent dye (Red Dye, Cole-Parmer, Vernon Hills, IL) was
added to all solutions at a rate of 0.5\% v v-1. The volumetric median
diameter (VMD) of droplets and volume percentage of droplets finer than
200 μm (V200) were measured at the Pesticide Application Technology
Laboratory using a Sympatec Helos-Vario K/R laser diffraction system
(Sympatec Inc., Clausthal, Germany), setup with a R7 lens, with a
dynamic size range of 9 to 1,800 μm. The distance from the nozzle tip to
the laser was 0.3 m. The VMD and V200 for each combination spray
solution versus nozzle type are listed in Table 1.

\hypertarget{application-and-field-layout}{%
\subsubsection{Application and field
layout}\label{application-and-field-layout}}

Prior to the applications, 27 drift collection stations were placed
downwind of the sprayed area in three transects (spaced by 7.6 m) and
perpendicular to the spray line (Figure 1). For each transect,
collectors were positioned at 1, 2, 3, 4, 8, 16, 31, 45, 60 m from the
edge of the application zone. Additionally, three drift stations were
placed 5 m upwind from the edge of the application zone and four petri
dishes (150 mm diameter) were placed in-swath. All collectors were
positioned 10 cm above the ground surface. Mylar cards (Grafix Plastics,
Cleveland, OH) were used as drift collectors. Cards with dimensions of
10 x 10 cm were placed upwind and up to 31 m downwind, whereas cards
with dimensions of 20 x 20 cm were placed at 45 and 60 m downwind.

Each replication was considered as one pass of the sprayer, equivalent
to 828 m2 in total area (91 m length x 9.1 m wide). Before each pass, a
new set of mylar cards and petri dishes were placed at the sampling
points. Five minutes after the end of each application was performed,
cards and petri dishes were collected and placed individually into
pre-labeled plastic bags. All samples were carefully managed to avoid
cross-contamination and stored into dark containers until further
analysis in laboratory in order to prevent photodegradation of rhodamine
dye. Samples were collected from the furthest to the nearest downwind
distance. Different teams were designated to work at downwind, upwind,
and in-swath zones, and gloves were changed between application passes.

The targeted wind velocity was between 3.6 to 6.7 m s-1 and ± 30° of
being perpendicular to the driveline before applying a treatment. When
necessary, the driveline and application zone was shifted to maintain
the ± 30° wind direction target. Meteorological conditions (air
temperature, and relative humidity, wind speed, wind direction) were
collected at 2 m height and 1-min intervals using a HOBO RX3000 Weather
Station (Onset Computer Co., Bourne, MA, USA) positioned upwind of the
sprayed area. The wind speed and direction data were collected using 2D
WindSonic anemometers (Gill Instruments, Lymington, UK). The
meteorological data for each respective treatment is listed in Figure 2.

\hypertarget{dye-quantification}{%
\subsubsection{Dye quantification}\label{dye-quantification}}

Samples were taken to the laboratory for dye extraction and
quantification using fluorometry technique. Distilled water was used as
extraction solution added to each bag using a bottle top dispenser
(LabSciences Inc., 60000-BTR, Reno, NV). Samples collected downwind were
rinsed with 50 mL of distilled water, whereas samples collected upwind
and in-swath were rinsed with 20 mL of distilled water. The samples were
agitated for 15 s and then a 1.5 mL aliquot from each sample bag was
drawn to fill a glass cuvette. The cuvette was placed in a rhodamine
module inside a fluorometer (Trilogy 7200.000, Turner Designs,
Sunnyvale, CA) using green light. Serial dilutions were performed upon
each tank sample to generate calibration curves, which allowed the
conversion of relative fluorescence unit into mg L-1 and further
expressing data into ɳL cm-2.

\hypertarget{greenhouse-study}{%
\subsection{Greenhouse study}\label{greenhouse-study}}

A completely randomized design

\hypertarget{statistical-analyses}{%
\subsection{Statistical analyses}\label{statistical-analyses}}

The statistical analyses were conducted with R statistical software
version 4.1.0.\textsuperscript{1} Data analyses were performed with
Bayesian inference with ``brms'' package.\textsuperscript{2} Bayesian
inference uses Markov chain Monte Carlo algorithms for sampling a
probability distribution;\textsuperscript{2} and avoids singular fit
from frequentist linear models when using complex random effects.

\hypertarget{field-study-1}{%
\subsubsection{Field study}\label{field-study-1}}

Solution, sprayer and nozzle factors were grouped as a single fixed
effect (herein treatments) due to missing factor water in Missouri.
Resulting in a combination of 12 treatments.

\hypertarget{spray-solution-deposition-at-upwind-and-inswath}{%
\paragraph{Spray solution deposition at upwind and
inswath}\label{spray-solution-deposition-at-upwind-and-inswath}}

Data was fitted to a mixed model using \emph{brm} function. Treatments
were the fixed effects and blocks nested within location random effects.
Model family was gaussian and prior distribution was set to student-t
with mean 0.5, standard deviation 3 and 11 degrees of freedom. The
posterior summaries (mean and highest posterior density) were estimated
with \emph{emmeans} function from the ``emmeans'' package. Treatment
means were compared using Bayes Factor (BF).\textsuperscript{3,4} In
short, if BF \textgreater{} 1 there is evidence for H1 (difference
between treatments); whereas, if H0 \textless{} 1 there is evidence for
H0 (no difference between treatments). If BF = 1, there is no evidence.
The level of evidence (anecdotal, moderate, strong, very strong, and
extreme) varies as the BF value increases (evidence for H1) or decreases
(evidence for H0).

\hypertarget{spray-solution-deposition-at-downwind}{%
\paragraph{Spray solution deposition at
downwind}\label{spray-solution-deposition-at-downwind}}

Data was fitted to a Bayesian linear mixed model using \emph{brm}
function. Spray solution deposition and distance were log-transformed to
meet linearity. A single model was fitted to each treatment. For each
model, treatments and distance were the fixed effects and blocks nested
within location random effects. Model family was gaussian and prior
distribution was set to student-t with mean 0.5, standard deviation 3
and 11 degrees of freedom. For clarification, intercepts, slopes were
back-transformed with \emph{exponential} function. Moreover, the linear
models fitted were used to predict the distance where no spray particle
deposition was detected (0 ηL cm\textsuperscript{-2}) for each
treatment, which was also back-transformed to m scale.

The area under the curve (AUC) was used to validate the linear models.
The spray solution deposition across distances within an experimental
unit were used to calculate the absolute AUC value. The AUC was
performed with \emph{audps} function from the ``agricolae'' package. The
AUC is commonly used for plant disease progress\textsuperscript{5,6} but
has been used to calculate herbicide injury.\textsuperscript{7} Data was
fitted to a mixed model using \emph{brm} function. Treatments were the
fixed effects and blocks nested within location random effects. Model
family was gaussian and prior distribution was set to student-t with
mean 0.5, standard deviation 3 and 11 degrees of freedom. The posterior
summaries and treatment means were estimated and compared using Bayes
Factor (BF) as above-mentioned.

\hypertarget{greenouse-study}{%
\subsection{Greenouse study}\label{greenouse-study}}

The Dv(10,50,90), RS, and \% drifable fines was fitted to a Bayesian
linear mixed model using \emph{brm} function. In the models, solution
and nozzle were set as fixed effects. Model family was hurdle gamma and
prior distribution was set to student-t with mean 0.5, standard
deviation 1 and 2 degrees of freedom. For each response variable, two
models were fitted: with and without interaction (solution and nozzle).
A model comparison was made with \textbf{bayesfactor\_models} from
``bayestestR'' package\textsuperscript{8} to investigate interaction
significance. For all response variables (Dvs, RS and \% drifable
fines), the best model was with interaction. The posterior summaries and
treatment means were estimated and compared using Bayes Factor (BF) as
above-mentioned.

\hypertarget{results}{%
\section{Results}\label{results}}

\hypertarget{spray-solution-deposition-at-inswath-and-upwind}{%
\subsection{Spray solution deposition at inswath and
upwind}\label{spray-solution-deposition-at-inswath-and-upwind}}

In general, Open sprayer treatments resulted in a more variable spray
particle deposition than Hood treatments (Figure 1). The inclusion of
either DRA or Water strongly impacted spray particle deposition inswath
for Open sprayer treatments, regardless nozzle type. The top and bottom
three treatments contained either DRA or Water, respectively. For
example, treatment DRA-Open-ULD resulted in the highest spray particle
deposition (1318.5 ηL cm\textsuperscript{-2}, Figure 1). In contrast,
911.2 nL cm\textsuperscript{-2} was the lowest spray solution
deposition, which was achieved with Water-Open-ULD treatment. Hood
sprayer treatments resulted in a more uniform spray particle deposition.
Furthermore, there were less than 0.29 ηL cm\textsuperscript{-2} spray
deposition at upwind with strong evidence (BF \textless{} 0.25) of no
difference between all pairwise treatment contrasts (data not shown).

\hypertarget{spray-solution-deposition-at-downwind-1}{%
\subsection{Spray solution deposition at
downwind}\label{spray-solution-deposition-at-downwind-1}}

Treatments with highest intercepts, which is the amount of spray
particle deposition near the treated area, were Water-Open-AIXR (15.7 ηL
cm\textsuperscript{-2}), followed by DRA-Open-AIXR (15.7 ηL
cm\textsuperscript{-2}), and Water-Open-ULD (12.0 ηL
cm\textsuperscript{-2}; Figure 2). In contrast, DRA-Hood-TTI,
DRA-Hood-ULD and DRA-Hood-AIXR treatments resulted in the lowest
intercepts (\textless{} 2.0 ηL cm\textsuperscript{-2}). In addition,
there is evidence that treatments with Hood sprayer provided faster
decay of spray particle deposition (slopes; Figure 2). The treatments
with highest slope decay were Water-Hood-TTI (-0.50), DRA-Hood-TTI
(-0.48), DRA-Hood-ULD (-0.44), Water-Hood-AIXR (-0.43), Water-Hood-ULD
(-0.43), and DRA-Hood-AIXR (-0.39).

The predicted distance where no spray particle deposition was detected
varied upon treatments (Figure 3A). In general, Open sprayer treatments
resulted in spray particle deposition at longest distances. For example,
the distance of non-detectable spray particle deposition with Open
sprayer treatments varied from 9.9 m (DRA-Open-TTI) to 54.9 m with
DRA-Open-AIXR; whereas Hood sprayer treatments varied from 1.4 to 8.2 m
with DRA-Hood-TTI and Water-Hood-AIXR, respectively.

Similar trend was observed in AUC. Treatments with Hood or Open sprayer
strongly impacted on AUC values (Figure 3B). The highest AUC values were
Water-Open-AIXR (87.4), followed by Water-Open-ULD (72.9) and
DRA-Open-AIXR (72.6). In contrast, DRA-Hood-TTI (13.9), DRA-Hood-ULD
(14.3) and Water-Open-AIXR (21.9) resulted in lowest AUC values. The
impact of Open and Hood sprayer is demonstrated in treatments including
AIXR nozzles. There is a high difference in AUC (50.7) between
DRA-Open-AIXR vs DRA-Hood-AIXR (BF \textgreater{} 100). Moreover,
addition of DRA did reduced AUC values when comparing within fixed
factors, sprayer (Hood and Open) and nozzle (AIXR, TTI and ULD);
however, addition of DRA were not statistically different for some
contrasts, including DRA-Hood-TTI vs Water-Hood-TTI (BF = 0.91),
DRA-Open-TTI vs Water-Open-TTI (BF = 0.55), and DRA-Open-AIXR vs
Water-Open-AIXR (BF = 0.93).

\hypertarget{dv-relative-span}{%
\subsection{Dv, Relative Span}\label{dv-relative-span}}

\hypertarget{discussion}{%
\section{Discussion}\label{discussion}}

\hypertarget{spray-solution-deposition-at-inswath-and-upwind-1}{%
\subsection{Spray solution deposition at inswath and
upwind}\label{spray-solution-deposition-at-inswath-and-upwind-1}}

\hypertarget{spray-solution-deposition-at-downwind-2}{%
\subsection{Spray solution deposition at
downwind}\label{spray-solution-deposition-at-downwind-2}}

\hypertarget{conclusion}{%
\section{Conclusion}\label{conclusion}}

\hypertarget{acknowledgments}{%
\section{Acknowledgments}\label{acknowledgments}}

\hypertarget{conflict-of-interest-declaration}{%
\section{Conflict of Interest
Declaration}\label{conflict-of-interest-declaration}}

\hypertarget{tables-each-table-complete-with-title-and-footnotes}{%
\section{Tables (each table complete with title and
footnotes)}\label{tables-each-table-complete-with-title-and-footnotes}}

\hypertarget{figure-legends}{%
\section{Figure Legends}\label{figure-legends}}

\hypertarget{references}{%
\section*{References}\label{references}}
\addcontentsline{toc}{section}{References}

\hypertarget{refs}{}
\begin{CSLReferences}{1}{0}
\leavevmode\hypertarget{ref-LanguageEnvironmentStatistical2021}{}%
\CSLLeftMargin{1 }
\CSLRightInline{R: {A Language} and {Environment} for {Statistical
Computing}, {R Foundation for Statistical Computing}, {Vienna, Austria}
(2021).}

\leavevmode\hypertarget{ref-burknerBrmsPackageBayesian2017}{}%
\CSLLeftMargin{2 }
\CSLRightInline{Bürkner P-C, \textbf{Brms} : {An} {\emph{R}} {Package}
for {Bayesian Multilevel Models Using} {\emph{Stan}}, \emph{J Stat Soft}
\textbf{80} (2017).}

\leavevmode\hypertarget{ref-leeBayesianCognitiveModeling2014}{}%
\CSLLeftMargin{3 }
\CSLRightInline{Lee MD and Wagenmakers E-J, Bayesian {Cognitive
Modeling}: {A Practical Course}, {Cambridge University Press} (2014).}

\leavevmode\hypertarget{ref-kassBayesFactors1995}{}%
\CSLLeftMargin{4 }
\CSLRightInline{Kass RE and Raftery AE, Bayes {Factors}, \emph{Journal
of the American Statistical Association} \textbf{90}:773--795,
{{[}American Statistical Association, Taylor \& Francis, Ltd.{]}}
(1995).}

\leavevmode\hypertarget{ref-meenaAreaDiseaseProgress2011}{}%
\CSLLeftMargin{5 }
\CSLRightInline{Meena PD, Chattopadhyay C, Meena SS, and Kumar A, Area
under disease progress curve and apparent infection rate of {Alternaria}
blight disease of {Indian} mustard ({Brassica} juncea) at different
plant age, \emph{Archives of Phytopathology and Plant Protection}
\textbf{44}:684--693, {Taylor \& Francis} (2011).}

\leavevmode\hypertarget{ref-simkoAreaDiseaseProgress2011}{}%
\CSLLeftMargin{6 }
\CSLRightInline{Simko I and Piepho H-P, The {Area Under} the {Disease
Progress Stairs}: {Calculation}, {Advantage}, and {Application},
\emph{Phytopathology®} \textbf{102}:381--389, {Scientific Societies}
(2011).}

\leavevmode\hypertarget{ref-striegelSpraySolutionPH2021}{}%
\CSLLeftMargin{7 }
\CSLRightInline{Striegel S, Oliveira MC, Arneson N, Conley SP,
Stoltenberg DE, and Werle R, Spray solution {pH} and soybean injury as
influenced by synthetic auxin formulation and spray additives,
\emph{Weed Technol} \textbf{35}:113--127 (2021).}

\leavevmode\hypertarget{ref-makowski2019}{}%
\CSLLeftMargin{8 }
\CSLRightInline{Makowski D, Ben-Shachar M, and Lüdecke D, {bayestestR}:
{Describing Effects} and their {Uncertainty}, {Existence} and
{Significance} within the {Bayesian Framework}, \emph{JOSS}
\textbf{4}:1541 (2019).}

\end{CSLReferences}

\end{document}
